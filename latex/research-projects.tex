\section{Research Projects}
\medskip

\begin{samepage}
\tlcventry{2023}{0}{\qibolab}{\githubsocialsymbol \ghurl{qiboteam/qibolab}}{}{}{
     Quantum hardware module and drivers for Qibo.
}
\end{samepage}

\begin{samepage}
\tlcventry{2023}{0}{\qibocal}{\githubsocialsymbol \ghurl{qiboteam/qibocal}}{}{}{
      Quantum calibration, characterization and validation module for Qibo.
}
\end{samepage}

\begin{samepage}
\tlcventry{2023}{0}{\qibo}{\githubsocialsymbol \ghurl{qiboteam/qibo}}{}{}{
      Quantum calibration, characterization and validation module for Qibo.
}
\end{samepage}


\subsection{\textbf{PhD}}

\begin{samepage}
\tlcventry{2019}{0}{\eko}{\githubsocialsymbol \ghurl{NNPDF/eko}}{}{}{
    Evolutionary Kernel Operator
    % The target was to create a new framework able to solve \dglap equations and
    % related problems (like the $\alpha_s$ running, required for PDF evolution).
    % \newline
    % In order to do this adopt more modern tools to improve
    % performances, we dedicated special care to specific case of the evolution
    % of a large number of PDFs (through reusability of the solution), and
    % lowered the barrier for newcomers and new feature implementations.
}
% \cvitem{\githubsocialsymbol}{\ghurl{N3PDF/eko}}
\end{samepage}

\begin{samepage}
\tlcventry{2019}{0}{\yadism}{\githubsocialsymbol \ghurl{NNPDF/yadism}}{}{}{
    Yet Another DIS Module
    % Inspired by the same principles of \eko, \yadism is providing a complete
    % library of analytical expression and tabulated numbers for the DIS
    % coefficient functions (including massive and intrinsic contributions).
    % \newline
    % It also give access to various other features, like Target Mass Corrections
    % (TMC), several Flavor Number Schemes (FNS), different definitions of
    % reduced cross-sections, and so on.
}
% \cvitem{\githubsocialsymbol}{\ghurl{N3PDF/yadism}}
\end{samepage}

\begin{samepage}
\tlcventry{2020}{2023}{\banana}{\githubsocialsymbol \ghurl{N3PDF/banana}}{}{}{
    pQCD benchmarking framework
    % This tool has been abstracted from the first benchmarks implemented for
    % \eko and \yadism.
    % \newline
    % In order to compare several different programs on a heavy complex space of
    % inputs (high rank, with several very non-trivial combinations) we needed
    % more than a couple of scripts.
    % That is why we defined a common interface to our and external program, and
    % we created a tool able to manage runs and the resulting comparisons.
    % \newline
    % Many different functions are available to keep track and compare these
    % results, plot them, and extract non-trivial information. All the output is
    % perfectly reproducible, since the complete information about it is stored
    % alongside the results in a benchmark database (based on SQLite).
}
% \cvitem{\githubsocialsymbol}{\ghurl{N3PDF/banana}}
\end{samepage}

\begin{samepage}
\tlcventry{2020}{0}{\pineappl}{\githubsocialsymbol \ghurl{NNPDF/pineappl}}{}{}{
    PineAPPL is not an extension of APPLgrid
    % Mostly contributed to the Python interface and packaging, in order to use
    % it as the backbone of the new PDF theory pipeline.
    % Development of the \eko integration, still for the same goal.
}
% \cvitem{\githubsocialsymbol}{\ghurl{N3PDF/pineappl}}
\end{samepage}

\begin{samepage}
\tlcventry{2021}{0}{\pinefarm}{\githubsocialsymbol \ghurl{NNPDF/pinefarm}}{}{}{
     Generate PineAPPL grids from PineCards 
}
% \cvitem{\githubsocialsymbol}{\ghurl{N3PDF/pineko}}
\end{samepage}

\begin{samepage}
\tlcventry{2021}{0}{\pineko}{\githubsocialsymbol \ghurl{NNPDF/pineko}}{}{}{
    \pineappl + \eko ─➤ fast theories
    % This package is the actual manager and handler used to compute the final
    % theory used for \pdf fitting.
    % I contributed to the design of the project, early development, review, and
    % the development of the CLI.
}
% \cvitem{\githubsocialsymbol}{\ghurl{N3PDF/pineko}}
\end{samepage}

\begin{samepage}
\tlcventry{2021}{2024}{\rr}{\githubsocialsymbol \ghurl{NNPDF/runcards}}{}{}{
    grid runcards for better reproducibility
    % Develop a runcard format and runner, to expose a unique interface to
    % several process calculation providers (Monte Carlo generators and non),
    % together with Christopher Schwan.
    % \newline
    % This project also (and arguably mainly) is based on the \pineappl format
    % and library, providing the process input for the rest of the theory
    % pipeline.
}
% \cvitem{\githubsocialsymbol}{\ghurl{NNPDF/runcards}}
\end{samepage}

\begin{samepage}
\tlcventry{2022}{2023}{\nnusf}{\githubsocialsymbol \ghurl{NNPDF/nnusf}}{}{}{
    neural interpolated neutrino structure functions
    % Provide a better data-driven model for low energy (i.e.\ $Q^2$) neutrino
    % structure functions, as an alternative to older models (e.g.\ Bodek-Yang).
}
% \cvitem{\githubsocialsymbol}{\ghurl{NNPDF/nnusf}}
\end{samepage}

\begin{samepage}
\tlcventry{2022}{0}{\nnpdf}{\githubsocialsymbol \ghurl{NNPDF/nnpdf}}{}{}{
    Proton's structure determination using contemporary methods of artificial
    intelligence.
    % Moderate contributions to the main \nnpdf project.
    % E.g.\ fixing and improving theory covariance matrix (th covmat) generation
    % and handling, or improving pseudo-data sampling in the presence of th covmat.
}
% \cvitem{\githubsocialsymbol}{\ghurl{NNPDF/nnpdf}}
\end{samepage}
\bigskip

\subsection{\textbf{Master}}

\begin{samepage}
\tlcventry{2018}{2020}{\cdt}{\githubsocialsymbol \ghurl{AleCandido/CDT\_2D}}{}{}{
    Causal Dynamical Triangulations with gauge fields, in two space-time dimensions.
    % We adopted the framework of Causal Dynamical Triangulations (CDT). 
    % It acts essentially as an ultraviolet regulator, defining the quantum
    % dynamics of gravity on a specific kind of piece-wise linear manifold, the
    % triangulation.
    % \newline
    % We introduced a gauge structure on these triangulations, generalizing the
    % pure gravity algorithm of CDT by the proper inclusion of operations defined
    % for static flat backgrounds, in a way independent on the dimension of
    % triangulations.
    % \newline
    % We also implement our algorithm and run the simulations, analyzing the
    % outputs and comparing them with the theoretical predictions.
}
% \cvitem{\githubsocialsymbol}{\ghurl{AleCandido/CDT\_2D}}
\end{samepage}
\bigskip
