\section{Theses}
\medskip

\subsection{\textbf{PhD Thesis}}

\cvitem{\textbf{Title}}{\textit{Theory Predictions for PDF Fitting}}
\cvitem{\textbf{Advisor}}{Prof. Stefano Forte}
\cvitem{\textbf{description}}{
    The overall target of the PhD has been to improve theory predictions for
    PDF fitting in several directions: \nnnlo (evolution and \dis),
    missing higher order uncertainties (\mhou, mainly based on scale
    variations), \qed evolution.\newline
    Moreover, the improvement has also been an improvement in performances, and
    in the structure of the code (as well as project management style and best
    practices). All the projects have been (and will be) developed open
    source.\newline
    Other side projects have become part of the thesis, related to the main
    software developments (and only possible because of them). Among these:
    evidence for intrinsic charm in \nnpdf4.0 (in the appropriate 3\fns),
    neural network interpolation of low $Q^2$ neutrino structure functions,
    and a study of Forward-Backward asymmetry, $A_{FB}$, in the Drell-Yan
    process.\newline
    A complementary study of an alternative \pdf extraction technique (based on
    a still very general parameterization, but with more analytical insight)
    has been initiated, and it is currently in its first phase.
}
\bigskip

\subsection{\textbf{MSc Thesis}}

\cvitem{\textbf{Title}}{\textit{Simplicial quantum gravity with dynamical gauge fields}}
\cvitem{\textbf{Advisor}}{Prof. Massimo D'Elia}
\cvitem{\textbf{description}}{
    We designed an algorithm for performing numerical simulations of the
    quantum dynamics of both gravity and gauge fields.
    It has been implemented, in two space-time dimensions, in an original code.
    Known analytical results (plus closely related ones, derived during the
    thesis itself) have been confirmed.\newline
    See \cdt below for more details on code and algorithm. Published in
    \cite{Candido:2020ybd}.
}
