\section{SM Theory Predictions}


\mhl{frameworks} The first part of my PhD project has been dedicated to develop
extensible frameworks, dedicated to solve specific problems (like \dglap
evolution and NNLO \dis predictions), collecting together decades of progresses
in perturbative \qcd calculations and making them available to the wider
community.

Similar applications were already existing, but they were lacking the purpose
of being open to the community, and to possible extensions in the most
straightforward way.
As explained above, this is the bigger target I aimed for with my collaborators
(these tools have also enabled the most precise to date determination of
intrinsic charm).
\bigskip

\mhl{interfaces} It is even better demonstrated by the latter part of PhD
project: the construction of a streamlined theory pipeline to produce efficient
theory predictions for \pdf fitting.

This is one of the best environment to showcase the abstract concepts expressed
before: doing a \pdf fit requires a lot of different inputs, from theory
(predictions for many processes, possibly resummation, and \pdf evolution) and
experiments (common interface to experimental data, and suitable covariances
intra- and inter-dataset).

Going more in depth on theory: many analytical calculations need to be
performed, and their results have to be collected inside programs.
It is rather limiting and naive to consider only the case in which a single
program is able to dispatch results for all processes at all available orders,
so to implement all the available calculations.

Thus, the more refined way to deal with the problem is not to enforce a single
provider, but to define a clean interface that different providers can adopt.
In the case of theory calculation, this goal is very well achieved by
interpolation grids, and in our case in particular by the \pineappl format, and
the same name library backing it.
\bigskip

Applying common interfaces leave enough freedom to experiment new solutions,
still taking part to community efforts. Without them, it would hard and
impossible to face the complex challenges of \hep, but imposing a single and
unique direction is not a viable alternative, since individual creativity is an
important ingredient in research.

An already successful effort in this direction has been the design of a common
format for \pdf exchange, \lhapdf. Different groups can extract \pdf{}s with very
different approaches, but a consumer can use any of them in the very same way.
