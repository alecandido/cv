\section{Quantum Gravity}

The synergy between software and physics is not limited to numerical
predictions and fits, but they can also extend to more abstract and theoretical
scenarios.

Indeed, my research interests are not limited to \pdf fits, and I have also
designed a Markov Chain Monte Carlo (\mcmc) algorithm on a space of curved
piece-wise flat manifold, with the addition lattice gauge fields (starting from
the known lattice-less algorithm).

Sampling distributions through a \mcmc is an expensive task, therefore it
requires a certain number of optimizations, and well designed representations.
Moreover, in order to obtain a suitable algorithm, several problems have to be
faced: from the generalization of \qcd concepts to a curved manifold, to the
complexity of the geometry with an increasing number of space-time dimensions.

This very different setup can also give insight on phenomena that have only
been observed in a flat background.
Some analytical and numerical limitations of the known algorithms can be
compared to the new results, and even the usual procedures can be reconsidered,
in order to solve common issues through new paths (e.g.\ introducing local
defects in flat lattices).
