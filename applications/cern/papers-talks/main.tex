\documentclass[10pt, a4paper, sans]{moderncv}

\usepackage{../../../latex/cv}
\usepackage{../../../latex/defs}

% Additional packages
\usepackage{multicol}
\usepackage{fancyhdr}

% Define lengths
\setlength{\columnsep}{1cm}
\setlength{\hintscolumnwidth}{10ex}

\newcommand{\journal}[4]{\textit{#1} \textbf{#2} (#3) #4}
\newcommand{\doi}[1]{\texttt{\href{https://doi.org/#1}{doi:#1}}}
\newcommand{\arxiv}[2]{\texttt{\href{https://arxiv.org/abs/#1}{arXiv:#1}} [#2]}
\newcommand{\tturl}[1]{\texttt{\url{#1}}}

\name{Alessandro}{Candido}
\title{Papers \& talks}

\begin{document}

\fancyfoot[LF]{
    \color{gray}
    \emailsymbol \emaillink{alessandro.candido@sns.it}
}
\fancyfoot[RF]{}

As requested by the job description, following are listed \enquote{Up to five
most important publications to which you have made a significant contribution},
\enquote{Up to five international conferences or workshops at which you have
given presentations}, and \enquote{Up to five public or internal notes to which
you have contributed personally}.

They are sorted from the most recent to the oldest.

\begin{multicols}{2}

\section{Publications}

\begin{enumerate}
    \item \textbf{Evidence for intrinsic charm quarks in the proton},
      R.~D.~Ball, A.~Candido, J.~Cruz-Martinez, S.~Forte, T.~Giani, F.~Hekhorn,
      K.~Kudashkin, G.~Magni, J.~Rojo, 
      \journal{Nature}{???}{???}{???},
      \arxiv{2208.???}{hep-ph},
      \doi{???}
      \newline
      Presenting evidence for intrinsic charm in the \nnpdf4.0 \pdf set,
      extracted using \eko and its backward evolution feature, that is able to
      evolve downward the \pdf in factorization scale, taking proper care of
      matching between different flavor number schemes, including intrinsic
      contributions.
    \item \textbf{Compact gauge fields on Causal Dynamical Triangulations: a 2D
      case study},
      A.~Candido, G.~Clemente, M.~D'Elia, F.~Rottoli,
      \journal{JHEP}{04}{2021}{184},
      \arxiv{2010.15714}{hep-lat},
      \doi{10.1007/JHEP04(2021)184} 
      \newline
      Designing a new algorithm to introduce lattice gauge fields in a
      dynamical space-time. The framework adopted to describe quantum nature of
      space-time is the Causal Dynamical Triangulations ({\small CDT}) one.
      The algorithm is general with respect to the gauge fields group and
      space-time dimension.
      A benchmark implementation is presented, in two dimension for $U(1)$ and
      $SU(2)$ gauge fields, which exhibit analytically predicted properties,
      and showcases possible directions to improve flat space lattice simulations.
    \item \textbf{Can $ \overline{\mathrm{MS}} $ parton distributions be negative?}
      A.~Candido, S.~Forte, F.~Hekhorn,
      \journal{JHEP}{11}{2020}{129},
      \arxiv{2006.07377}{hep-ph},
      \doi{10.1007/JHEP11(2020)129}
      \newline
      Discussing the problem of \pdf positivity in $\overline{MS}$ scheme. In
      order to investigate it, the behavior of the partonic cross-sections is
      analyzed, together with the change of scheme from a factorization scheme
      in which \pdf{}s are positive by definition (because linked to physical
      observables).
\end{enumerate}

\columnbreak

\section{Conferences}

\begin{enumerate}
    \item \textbf{Machine Learning @ GGI}
      \nnpdf{} and beyond: machine learning and statistics for PDF extraction,
      \tturl{https://www.ggi.infn.it/showevent.pl?id=414}
      The \nnpdf{4.0} methodology makes use of many ideas coming from the
      machine learning community, improving the flexibility and reducing the
      assumptions on the unknown set of functions.
      Further developments can exploit different statistics approaches, to
      enhance the transparency of the underlying functional prior.
    \item \textbf{ISMD 2022},
      Theory Predictions for PDF fitting,
      \tturl{https://indico.cern.ch/event/1015549/contributions/4903597/}
      \newline
      A further presentation of the theory pipeline, including \eko, \yadism,
      and \pineappl, to a somewhat broader audience, since ISMD gather people
      from different fields. The \textit{evidence for intrinsic charm} has also
      been presented.
    \item \textbf{ICHEP 2022}, 
      EKO and yadism: theory predictions for PDF fitting,
      \tturl{https://agenda.infn.it/event/28874/contributions/169937/}
      \newline
      \eko and \yadism have been presented to the general public of high energy
      physicists, together with the whole pipeline they are integrated in, as a
      way to compute streamlined fast theory predictions, with the main target
      of \pdf fits and other analyses (but the individual tools can do more).
      The \textit{evidence for intrinsic charm} in the proton is presented as
      an early example of the success of these tools to produce original
      physics.
\end{enumerate}

\vspace{30pt}
\section{Notes}

\begin{enumerate}
    \item \textbf{MHOU prescriptions}
      \newline
      An in-depth derivation and discussion of prescriptions, for the estimate
      of Missing Higher Order Uncertainties based on scale variations.
    \item \textbf{\yadism: documentation}
      \newline
      \tturl{https://yadism.readthedocs.io/}
      \newline
      Contains both the technical documentation about the piece of code and its
      usage, but also a collection and review of the underlying theory (mainly
      oriented to the prospective user).
    \item \textbf{\eko: documentation}
      \newline
      \tturl{https://eko.readthedocs.io/}
      \newline
      Contains both the technical documentation about the piece of code and its
      usage, but also a collection and review of the underlying theory (mainly
      oriented to the prospective user).
\end{enumerate}


\end{multicols}

\end{document}

